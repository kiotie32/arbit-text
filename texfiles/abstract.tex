In today’s world of changing technologies due to advancement of technology,more and more organizations are shifting to usage off cloud computing services worldwide. With cloud computing service provider companies like Google, Amazon, AliBaba, Microsoft ,IBM  investing more and more in cloud computing.It is becoming target of internet threats, such as malware or virus, technical vulnerability and negligent behaviours.
Hence anomaly detection in cloud has been an emerging area of research for researchers like us. This thesis  addresses the main security and privacy issues in Cloud Computing. It is essential to have an Anomaly Detection System (ADS) to detect anomalies with a high detection accuracy in cloud environment. With the evolving technological setup this work becomes complex day by day. This work proposes an anomaly detection system at the administrator level to improve the accuracy of the detection system. 
The proposed system is implemented and it uses classifiers and data mining techniques to detect anomalies in network. The DARPA’s KDD cup data-set 1999 is used for experiments.  Using simple data analytics techniques the system is able to find out anomalies in cloud.
Machine Learning has various  applications: classification, predicting next value, anomaly detection, and discovering structure. In this thesis we have studied,how anomaly detection detects anomalies from security perspective on cloud service providers. Anomaly detection has a wide range of applications such as fraud detection, surveillance, diagnosis, data cleanup, and predictive maintenance.